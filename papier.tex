%% 
%% Copyright 2007, 2008, 2009 Elsevier Ltd
%% 
%% This file is part of the 'Elsarticle Bundle'.
%% ---------------------------------------------
%% 
%% It may be distributed under the conditions of the LaTeX Project Public
%% License, either version 1.2 of this license or (at your option) any
%% later version.  The latest version of this license is in
%%    http://www.latex-project.org/lppl.txt
%% and version 1.2 or later is part of all distributions of LaTeX
%% version 1999/12/01 or later.
%% 
%% The list of all files belonging to the 'Elsarticle Bundle' is
%% given in the file `manifest.txt'.
%% 

%% Template article for Elsevier's document class `elsarticle'
%% with numbered style bibliographic references
%% SP 2008/03/01

%% \documentclass[preprint,12pt]{elsarticle}

%% Use the option review to obtain double line spacing
%% \documentclass[authoryear,preprint,review,12pt]{elsarticle}

%% Use the options 1p,twocolumn; 3p; 3p,twocolumn; 5p; or 5p,twocolumn
%% for a journal layout:
%% \documentclass[final,1p,times]{elsarticle}
%% \documentclass[final,1p,times,twocolumn]{elsarticle}
%% \documentclass[final,3p,times]{elsarticle}
\documentclass[final,3p,times,twocolumn]{elsarticle}
%% \documentclass[final,5p,times]{elsarticle}
%% \documentclass[final,5p,times,twocolumn]{elsarticle}

%% For including figures, graphicx.sty has been loaded in
%% elsarticle.cls. If you prefer to use the old commands
%% please give \usepackage{epsfig}

%% The amssymb package provides various useful mathematical symbols
\usepackage{amssymb}
%% The amsthm package provides extended theorem environments
%% \usepackage{amsthm}

%% The lineno packages adds line numbers. Start line numbering with
%% \begin{linenumbers}, end it with \end{linenumbers}. Or switch it on
%% for the whole article with \linenumbers.
\usepackage{lineno}

\journal{Computational and Theoretical Chemistry}

\usepackage{lipsum}
\graphicspath{{img/}}
\newcommand{\trans}{\emph{trans}-disiloxybenzocyclobutene}
\begin{document}

\begin{frontmatter}

%% Title, authors and addresses

%% use the tnoteref command within \title for footnotes;
%% use the tnotetext command for theassociated footnote;
%% use the fnref command within \author or \address for footnotes;
%% use the fntext command for theassociated footnote;
%% use the corref command within \author for corresponding author footnotes;
%% use the cortext command for theassociated footnote;
%% use the ead command for the email address,
%% and the form \ead[url] for the home page:
%% \title{Title\tnoteref{label1}}
%% \tnotetext[label1]{}
%% \author{Name\corref{cor1}\fnref{label2}}
%% \ead{email address}
%% \ead[url]{home page}
%% \fntext[label2]{}
%% \cortext[cor1]{}
%% \address{Address\fnref{label3}}
%% \fntext[label3]{}

\title{Influence of heteroatoms on the opening of benzocyclobutene}

%% use optional labels to link authors explicitly to addresses:
\author{Bertille Castel}
\author{Jean-Marc Mattalia}
\author{Paola Nava}
\author{Yannick Carissan}
\address{Aix Marseille Universit\'e, Centrale Marseille, CNRS,
iSm2 UMR 7313,
13397, Marseille, France}

\begin{abstract}
%% Text of abstract
Abstract
\end{abstract}

\begin{keyword}
%% keywords here, in the form: keyword \sep keyword

%% PACS codes here, in the form: \PACS code \sep code

%% MSC codes here, in the form: \MSC code \sep code
%% or \MSC[2008] code \sep code (2000 is the default)
Benzocyclobutene \sep DFT \sep retro Diels Alder
\end{keyword}

\end{frontmatter}

\linenumbers

%% main text
\section{Introduction}
\label{sec:introduction}
Recent combined experimental and theoretical work
shows that \trans{}, a benzocyclobutene
derivative,
reacts easily with dioxygen in its triplet state.\cite{Drujon2014}
In this later article, it is demonstrated that
the reactivity of benzocyclobutene derivatives can only be
explained by a diradical electronic structure.
The diradical character of singlet state "closed shell" systems
draws a lot of attention nowadays~\cite{Abe2013,Hsieh2015}
mainly due to the fact that it does not respect the Lewis electron
coupling rule.\cite{Trinquier2015}

Later on, some of us showed the influence of the combination
of ring strain of the cyclobutene ring and aromaticity of the
benzene ring on the reactivity of such species.\cite{Nava2014a}

The control of the diradical character of the wavefunction
of a molecule leads to a huge number of applications 
in a broad range of fields from solar energy harvesting
(charge separation induced by light induces a current)
to the detection of radical species in living species
(detection of NO in the blood of a patient is indicative
of a future heart attack).

In this article, we focus on the influence of the introduction
of heteroatoms in benzocyclobutene on the opening
of the cyclobutene ring, Fig.\ref{fig:reaction}.
\section{Reactivity with $O_2$}
The most remarkable feature of 
\section{Computational details}
\label{sec:computational_details}
All calculations were done in the framework of density
functional theory with the TURBOMOLE program package.
The PBE0 functional is used with the def2-TZVP basis
set.\cite{pbe0,def2tzvp}
This showed to give the better agreement with highly correlated
methods in a previous study.\cite{Nava2014a}
Except if explicitly notified, reactant and product geometries
were found to be minima of the potential energy surfaces
by calculation of the hessian.
For transition states, except if explicitly notified, one and
only one negative eigenvalue
of the hessian matrix was found along the reaction coordinate.
\section{Results and discussion}
\label{sec:results_and_discussion}
The influence on the reaction enthalpy of the opening reaction
of substitution on carbon atoms belonging to
the six membered ring of \trans\ was tested.
Results are gathered in Table~\ref{tab:reactions_lig}.
The influence on the same reaction of the replacement
of C-H groups by nitrogen atoms was also studied.
These results can be found in Table~\ref{tab:reactions_N}.
\section{Conclusion}
\label{sec:conclusion}

%% The Appendices part is started with the command \appendix;
%% appendix sections are then done as normal sections
%% \appendix

%% \section{}
%% \label{}

%% If you have bibdatabase file and want bibtex to generate the
%% bibitems, please use
%%
\bibliographystyle{elsarticle-num} 
\bibliography{biblio}

%% else use the following coding to input the bibitems directly in the
%% TeX file.

%% \begin{thebibliography}{00}

%% \bibitem{label}
%% Text of bibliographic item

%% \bibitem{}

%% \end{thebibliography}
\newpage
\input{reactions.tex}
\newpage
\newpage
\input{fig_reaction.tex}
\newpage
\end{document}
\endinput
%%
%% End of file `elsarticle-template-num.tex'.
